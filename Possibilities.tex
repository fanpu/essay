%%%%%%%%%%%%%%%%%%%%%%%%%%%%%%%%%%%%%%%%%
% Formal Text-Rich Title Page 
% LaTeX Template
% Version 1.0 (27/12/12)
%
% This template has been downloaded from:
% http://www.LaTeXTemplates.com
%
% Original author:
% Peter Wilson (herries.press@earthlink.net)
%
% License:
% CC BY-NC-SA 3.0 (http://creativecommons.org/licenses/by-nc-sa/3.0/)
% 
% Instructions for using this template:
% This title page compiles as is. If you wish to include this title page in 
% another document, you will need to copy everything before 
% \begin{document} into the preamble of your document. The title page is
% then included using \titleGP within your document.
%
%%%%%%%%%%%%%%%%%%%%%%%%%%%%%%%%%%%%%%%%%

%----------------------------------------------------------------------------------------
%	PACKAGES AND OTHER DOCUMENT CONFIGURATIONS
%----------------------------------------------------------------------------------------

\documentclass{book}
\usepackage{graphicx}
\usepackage{dirtytalk}
\usepackage{fancyhdr}
\usepackage{wrapfig}
\usepackage[normalem]{ulem}
\usepackage[utf8]{inputenc}
\usepackage[T1]{fontenc}
\usepackage[table]{xcolor}

\pagestyle{fancy}
\fancyhead[LE,RO]{}
\fancyhead[RE,LO]{\rightmark}
\fancyfoot[CE,CO]{\leftmark}
\fancyfoot[LE,RO]{\thepage}
\fancyfoot[RE,LO]{ZENG FAN PU}
\renewcommand{\headrulewidth}{2pt}
\renewcommand{\footrulewidth}{1pt}




\newcommand*{\plogo}{\fbox{$\mathcal{VTX}$}} % Generic publisher logo

%----------------------------------------------------------------------------------------
%	TITLE PAGE
%----------------------------------------------------------------------------------------

\newcommand*{\titleGP}{\begingroup % Create the command for including the title page in the document
\centering % Center all text
%\vspace*{\baselineskip} % White space at the top of the page

\rule{\textwidth}{1.6pt}\vspace*{-\baselineskip}\vspace*{2pt} % Thick horizontal line
\rule{\textwidth}{0.4pt}\\[\baselineskip] % Thin horizontal line

 {\Huge Possibilités Infinies}\\ {\Large ESSAYS FOR 2016 US APPLIATION \\[0.3\baselineskip] ``Failure is Not An Option!''}\\[0.2\baselineskip] % Title

\rule{\textwidth}{0.4pt}\vspace*{-\baselineskip}\vspace{3.2pt} % Thin horizontal line
\rule{\textwidth}{1.6pt}\\[\baselineskip] % Thick horizontal line

\scshape % Small caps-0
Striking into the core of my identity, \\ % Tagline(s) or further description
Asking myself once more, ``Who am I?'' \\[\baselineskip] % Tagline(s) or further description

\vspace*{5\baselineskip} % Whitespace between location/year and editors

Written by \\[\baselineskip]
{\Large ZENG FAN PU\par} % Editor list
{\itshape Hwa Chong Institution \\ Singapore\par} % Editor affiliation

\vspace*{\baselineskip} % Whitespace between location/year and editors
\includegraphics[scale=0.15]{Stanford}

\vfill % Whitespace between editor names and publisher logo

\plogo \\[0.3\baselineskip] % Publisher logo
{\scshape 2016} \\[0.3\baselineskip] % Year published
{\large VORTEX PUBLISHINGS}\par % Publisher

\endgroup}

%----------------------------------------------------------------------------------------
%	BLANK DOCUMENT
%----------------------------------------------------------------------------------------

\begin{document} 
\thispagestyle{empty}
%\pagestyle{empty} % Removes page numbers

\titleGP % This command includes the title page
\pagenumbering{arabic}
\setcounter{tocdepth}{4}

\tableofcontents
\chapter{Personal Statement}
\section{2016-2017 Essay Prompts}
\begin{enumerate}
	\item Some students have a background, identity, interest, or talent that is so meaningful they believe their application would be incomplete without it. If this sounds like you, then please share your story.


	\item The lessons we take from failure can be fundamental to later success. Recount an incident or time when you experienced failure. How did it affect you, and what did you learn from the experience?


	\item Reflect on a time when you challenged a belief or idea. What prompted you to act? Would you make the same decision again?


	\item Describe a problem you've solved or a problem you'd like to solve. It can be an intellectual challenge, a research query, an ethical dilemma - anything that is of personal importance, no matter the scale. Explain its significance to you and what steps you took or could be taken to identify a solution.


	\item Discuss an accomplishment or event, formal or informal, that marked your transition from childhood to adulthood within your culture, community, or family.
\end{enumerate}

\section{Prompt 1}
\textbf{Some students have a background, identity, interest, or talent that is so meaningful they believe their application would be incomplete without it. If this sounds like you, then please share your story.}

\subsection{Sketchpad}
I wrote this last year. There's that unmistakable tinge whenever I see this question because I keep feeling the bite of getting rejected last year, there's no mistaking that. Well, what can I say....I believe it's imperative that I put the past behind me and proceed without any sort of hindsight bias? Just because I got rejected by writing about entrepreneurship doesn't mean that the entire topic is off-limits. Heck, I have no idea what even goes on in the minds of the examiners. At any rate, while I should appreciate the feedback that perhaps my essay is not up to the stringent standard as desired by such exceptional schools, I should not entirely dismiss my entire entrepreneurial experience and journey and simply put it behind me...this time now, offers yet another checkpoint for me to reconsolidate, regroup and get something better out again.

\section{Prompt 2}
\textbf{The lessons we take from failure can be fundamental to later success. Recount an incident or time when you experienced failure. How did it affect you, and what did you learn from the experience?}

\subsection{Sketchpad}
Well there's many incidents in which I failed...but those that I remember most deeply, those those lesson I'll carry through the longest are undeniably that of my startups. I also failed in competitions and studies too although that wasn't as impactful as startups. Sigh. Startups...will it be too cliche though and would it be better if I actually addressed it for the first question instead?

\section{Prompt 3}
\textbf{Reflect on a time when you challenged a belief or idea. What prompted you to act? Would you make the same decision again?}
\end{document}